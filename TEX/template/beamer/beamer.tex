% Copyright 2004 by Till Tantau <tantau@users.sourceforge.net>.
%
% In principle, this file can be redistributed and/or modified under
% the terms of the GNU Public License, version 2.
%
% However, this file is supposed to be a template to be modified
% for your own needs. For this reason, if you use this file as a
% template and not specifically distribute it as part of a another
% package/program, I grant the extra permission to freely copy and
% modify this file as you see fit and even to delete this copyright
% notice. 

\documentclass{beamer}
\usepackage{ctex}
\usepackage{clrscode3e}
% There are many different themes available for Beamer. A comprehensive
% list with examples is given here:
% http://deic.uab.es/~iblanes/beamer_gallery/index_by_theme.html
% You can uncomment the themes below if you would like to use a different
% one:
%\usetheme{AnnArbor}
%\usetheme{Antibes}
%\usetheme{Bergen}
%\usetheme{Berkeley}
%\usetheme{Berlin}
%\usetheme{Boadilla}
%\usetheme{boxes}
%\usetheme{CambridgeUS}
%\usetheme{Copenhagen}
%\usetheme{Darmstadt}
%\usetheme{default}
%\usetheme{Frankfurt}
%\usetheme{Goettingen}
%\usetheme{Hannover}
%\usetheme{Ilmenau}
%\usetheme{JuanLesPins}
%\usetheme{Luebeck}
\usetheme{Madrid}
%\usetheme{Malmoe}
%\usetheme{Marburg}
%\usetheme{Montpellier}
%\usetheme{PaloAlto}
%\usetheme{Pittsburgh}
%\usetheme{Rochester}
%\usetheme{Singapore}
%\usetheme{Szeged}
%\usetheme{Warsaw}

\title{LDA概率模型简介}

% A subtitle is optional and this may be deleted
\subtitle{Latent Dirichlet Allocation}

\author{李修成\\12S003016}
%\author{F.~Author\inst{1} \and S.~Another\inst{2}}

\institute[HIT]
{
    Department of Computer Science\\
    Harbin Institute of Technology
}

%\institute[Universities of Somewhere and Elsewhere] % (optional, but mostly needed)
%{
%  \inst{1}%
%  Department of Computer Science\\
%  University of Somewhere
%  \and
%  \inst{2}%
%  Department of Theoretical Philosophy\\
%  University of Elsewhere
%}

\date{}
%\date{Conference Name, 2013}

\subject{Theoretical Computer Science}

%\AtBeginSubsection[]
%{
%  \begin{frame}<beamer>{Outline}
 %   \tableofcontents[currentsection,currentsubsection]
%  \end{frame}
%}

% Let's get started
\setlength{\parindent}{0pt}
\begin{document}

\begin{frame}
  \titlepage
\end{frame}

%\begin{frame}{Outline}
%  \tableofcontents
%\end{frame}


\begin{frame}
  \begin{itemize}
  \item {
        LDA(Latent Dirichlet Allocation), a generative probabilistic model for collections of discrete data such as
        text corpora and has been widely used in the area like text model, content-based image retrievel and 
        bioinformatics.
  }
  \item LDA was developed by David Blei when he pursued his PhD in UC Berkeley learning from M. Jordan.\\
  \end{itemize}
\end{frame}

\begin{frame}
 \begin{itemize}
  \item {
    Essentially, LDA can be seen as an effective method of reduction like PCA which could find the main features
    of the data. From the history perspective, it is also the further development of pLSI(but we do not discuss 
    it today).
  }
  \item {
        As we present above, LDA could be used in many different areas but for convenience we would like
        to discuss it using the example of text modeling.
  }  
  \end{itemize}
\end{frame}

\begin{frame}{Mathematica Supplementary}
    \begin{itemize}
        \item {
                Multinominal distribution, 
           \begin{equation}
               \begin{split}
                Multi(m_1,m_2,\ldots,m_k|\theta, N)=&
                \frac{N!}{m_1!m_2! \ldots m_k!}  \theta_1^{m_1} \ldots \theta_k^{m_k} , \\
                & (\Sigma _{i=1}^k m_i=N,  \Sigma _{i=1}^k \alpha _i=1)
                \end{split} 
           \end{equation}
            }
        \item {
                Dirichlet distribution,
            \begin{equation}
                 Dir( \theta | \alpha )=
                \frac{ \Gamma ( \Sigma _{i=1}^k  \alpha _i)}{ \Pi _{i=1}^k \Gamma (\alpha_i)}
                \theta _1^{\alpha_1-1} \ldots \theta _k^{\alpha_k-1}
            \end{equation}
            }
        \item  {
                \begin{equation}
                 \Gamma (x) = \int_{0}^{ \infty } u^{x-1}e^{-u} \, du,
                \end{equation}
        }
        其中,$\Gamma (x+1) = x \Gamma (x), \Gamma(1) = 1$,$\Gamma(x)$为阶乘函数$n!$在实数集上的延拓
        \item {
        The Dirichlet is conjugate to the multinomial distribution.
        }
    \end{itemize}
\end{frame}


\begin{frame}{Notation and  terminology}
    \begin{itemize}
        \item word(pixel),Vocabulary indexed by$\{1,\ldots,V\}$
        \item document(image),$\textbf{w} = (w_1,w_2,\ldots, w_N)$
        \item corpus(collections of image), $D = \{ \textbf{w}_1, \textbf{w}_2, \ldots, \textbf{w}_M \}$
    \end{itemize}
\end{frame}



\begin{frame}
    \begin{itemize}
    \item {
    The basic idea is that documents are represented as random mixtures over latent topics, where each topic
    is characterized by a distribution over words.
    }

    \item {
    LDA assumes the following generative process for each document $\textbf{w}$ in a corpus $D$:
    \begin{codebox}
        \li Choose $N \sim Poisson$.
        \li Choose $\theta \sim Dir(\alpha)$.
        \li \For each of the N words $w_n$:
        \li    \Do
                Choose a topic $z_n \sim Multinominal(\theta)$.
        \li     Choose a word $w_n$ from $p(w_n|z_n,\beta)$, a multinomial
        \li        probability conditioned on the topic $z_n$.
            \End
    \end{codebox}

    $\beta_{k \times V}$ is a $k \times V$ matrix, $\beta_{ij} = p(w^j=1|z^i=1)$
    }
    \end{itemize} 

\end{frame}

\begin{frame}{LDA presenting in PGM}
\begin{figure}[htbp]
	\centerline{\includegraphics[width=0.8\textwidth]{lda_plate.png}}
	\caption[]{\label{c} Three-level hierarchical Bayesian model}
\end{figure}
\end{frame}


\begin{frame}
\begin{itemize}
	\item {
	Given the parameters $\alpha$ and $\beta$, the joint distribution of a topic
	mixture $\theta$, a set of $N$ topics $\textbf{z}$, and a set of $N$ words 
	$\textbf{w}$ is given by:
	\begin{equation}
	p( \theta ,\textbf{z},\textbf{w}| \alpha , \beta ) = 
	p( \theta | \alpha ) \Pi _{n=1}^Np(z_n| \theta )p(w_n|z_n, \beta )
	\end{equation}
	}
	\item {
	\begin{equation}
	p( \textbf{w}| \alpha , \beta ) = \int
	p( \theta | \alpha ) (\Pi _{n=1}^N  \Sigma_{z_n}  p(z_n| \theta )p(w_n|z_n,  \beta ))\ d\theta
	\end{equation}
	}
	\item {
	posterior probability
	\begin{equation}
	p(\theta,\textbf{z}|\textbf{w},\alpha,\beta) = 
	\frac{p( \theta ,\textbf{z},\textbf{w}| \alpha , \beta )}
	{p( \textbf{w}| \alpha , \beta )}
	\end{equation}
	}
\end{itemize}
\end{frame}

\begin{frame}{Parameter Inference}
\begin{itemize}

\item {
\begin{equation}
p(\textbf{w}| \alpha , \beta )= \frac{ \Gamma(\Sigma _i \alpha _i)}{ \Pi _i \Gamma ( \alpha _i)} \int 
( \Pi _{i=1}^k \theta_i^{\alpha_i-1} )( \Pi_{n=1}^N  \Sigma_{i=1}^k  \Pi_{j=1}^V( \theta _i \beta _{ij})^{w_n^j} ) d\theta
\end{equation} 
However, it is intractable (Dickey,1983)!!
}
\pause

	\item Variational approximation
	\item Markov chain Monte Carlo
\end{itemize}
\end{frame}

\begin{frame}
	\begin{example}
	We choose three articles from wikipedia:
	\begin{enumerate}
		\item Jay Chou
		\item Computer Science
		\item Finance
\end{enumerate}	 
	\end{example}
\end{frame}

\begin{frame}{K Topics}
\begin{figure}[htbp]
	\centerline{\includegraphics[width=0.8\textwidth]{lda_words.png}}
	%\caption[]{\label{a} Topics}
\end{figure}
\end{frame}

\begin{frame}{$\Theta$ in each docment}
\begin{figure}[htbp]
	\centerline{\includegraphics[width=0.8\textwidth]{lda_theta.png}}
	%\caption[]{\label{b} $\Theta$ in each docment}
\end{figure}

\begin{columns}[t] 

	\begin{column}[T]{5cm} 
          \begin{enumerate}
     		\item Doc: Jay Chou 
			\item Doc: Computer Science
			\item Doc: Finance
     	  \end{enumerate}
     \end{column}
     \pause
     \begin{column}[T]{5cm} % each column can also be its own environment
     	\begin{enumerate}
     		\item Topic: Computer Science  
			\item Topic: Finance 
			\item Topic: Jay Chou 
     	\end{enumerate}
     \end{column}
     
     
\end{columns}

\end{frame}


\begin{frame}{Blocks}
	\begin{block}{Block Title}
		You can also highlight sections of your presentation in a block, with it's own title
	\end{block}

	\begin{theorem}
		There are separate environments for theorems, examples, definitions and proofs.
	\end{theorem}

	\begin{example}
		Here is an example of an example block.
	\end{example}
\end{frame}


\end{document}


